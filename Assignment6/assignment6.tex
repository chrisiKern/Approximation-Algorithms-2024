\documentclass{article}
\usepackage{graphicx}
\usepackage{amsmath, amsfonts, mathtools}
\usepackage{amsthm}
\usepackage{todonotes}
\usepackage[ruled, linesnumbered]{algorithm2e}
\usepackage{enumerate}
\usepackage{float}
\usepackage{subcaption}
\usepackage{dsfont}
\usepackage{pythonhighlight}
\usepackage{listings}
\usepackage{xcolor}
\usepackage{xurl}
\usepackage{algorithm2e}
\usepackage{hyperref}
\usepackage[numbers]{natbib}

% Define colors for syntax highlighting
\definecolor{codegreen}{rgb}{0.2,0.6,0.2}
\definecolor{codegray}{rgb}{0.5,0.5,0.5}
\definecolor{codepurple}{rgb}{0.58,0,0.82}
\definecolor{backcolour}{rgb}{0.95,0.95,0.92}


% Sensible defaults for lstlistings
\lstset{
numberstyle=\tiny\color{codegray},
backgroundcolor=\color{backcolour},   
  basicstyle=\footnotesize\ttfamily,
  belowcaptionskip=1\baselineskip,
  breaklines=true,
  commentstyle=\bfseries\color{purple!40!black}
  frame=L,
  identifierstyle=\color{blue},
  keywordstyle=\bfseries\color{green!40!black},
  language=python,
  showstringspaces=false,
  stringstyle=\color{orange},
  xleftmargin=\parindent,
  captionpos=b,                    
    keepspaces=true,                 
    numbers=left,                    
    numbersep=5pt,                  
    showspaces=false,                
    showstringspaces=false,
    showtabs=false,
    tabsize=2,
    framexleftmargin=16pt,
    framextopmargin=6pt,
        framexbottommargin=6pt, 
        frame=tb, framerule=0pt,
}


\title{Approximation Algorithms - Assignment 6}
\author{Group 2: Christoph Kern, Johannes Gabriel Sindlinger}


\begin{document}

\maketitle

\begin{center}
    \textbf{Github-Link:} \\
\url{https://github.com/chrisiKern/Approximation-Algorithms-2024/blob/main/Assignment6/Correlation_clustering_assignment.ipynb}
\end{center}

For this assignment, we are considering the \emph{correlation clustering} problem, defined as follows:
\begin{quote}
    \begin{description}
        \item[Input:] Undirected graph $G=(V,E)$ with $|V| = n$, two edge weights $w_{ij}^+, w_{ij}^- \ge 0$ for each  $\{i,j\} \in E$.
        \item[Goal:] A partition $\mathcal{P} = P_1 \cup \dots \cup P_k$ of $V$ which maximizes the total $w^+$ weight of edges inside the sets of the partition plus the total $w^-$ weight of edges between the sets of the partition, i.e.
    \end{description}
\end{quote}
 \[
    \sum_{\{i,j\} \in E(\mathcal{P})} w_{ij}^+ + \sum_{\{i,j\} \in E \setminus E(\mathcal{P})} w_{ij}^- \quad \text{ with } E(\mathcal{P}) = E \cap \bigcup\limits_{P_\ell \in \mathcal{P}} \{\{i,j\} \subseteq P_\ell \}
\]
\vspace{1em}

The following vector program coincides with the relaxation of the correlation clustering problem (as found in section 6.4 in the book):

\begin{align*}
    \text{maximize }\sum_{\{i,j\} \in E} \left(w_{ij}^+(v_i\cdot v_j) + w_{ij}^-(1-v_i\cdot v_j) \right)\\
    \text{s.t. }\quad\quad v_i\cdot v_i &= 1 & \forall i \in V\\
    v_i\cdot v_j &\ge 0 & \forall i,j \in V\\
    v_i &\in \mathbb{R}^n & \forall i \in V
\end{align*}

This problem can be equivalently expressed as following SDP:

\begin{align*}
    \text{maximize }\sum_{\{i,j\} \in E} \left(w_{ij}^+x_{ij} + w_{ij}^-(1-x_{ij}) \right)\\
    \text{s.t. }\quad\quad x_{ii} &= 1 & \forall i \in V\\
    x_{ij} &\ge 0 & \forall i,j \in V\\
    X &\succeq 0
\end{align*}

In python, we construct and solve the above SDP as follows:
\begin{lstlisting}
correlation_clustering = pc.Problem()

X = pc.SymmetricVariable('X',(n,n))
W_p = pc.Constant('W_p', W_plus)
W_m = pc.Constant('W_m', W_minus)
ones = pc.Constant('ones', np.ones((n,n)))

correlation_clustering.set_objective('max', (W_p | X) + (W_m | (ones - X)))
correlation_clustering.add_constraint(pc.maindiag(X) == 1)
correlation_clustering.add_constraint(X >= 0)
correlation_clustering.add_constraint(X >> 0)

correlation_clustering.solve(solver='cvxopt')
V = cholesky(X.value + 1e-3 * np.identity(n))
\end{lstlisting}

\begin{itemize}
    \item Line 1: Initialize optimization problem
    \item Lines 3-6: Define variables and constants
    \item Lines 8-11: Set objective and add constraints
    \item Lines 13-14: Solve the SDP and extract vectors of the equivalent vector program formulation.
\end{itemize}


In code, we can determine the upper bound of OPT by reading the objective function value of the SDP:
\begin{lstlisting}
correlation_clustering.value
\end{lstlisting}
For the instance \texttt{denmark-0.6.txt}, we obtain an upper bound of $1629.86$.\vspace{1em}

To compute the expected value of the objective function with a uniformly random 4-clustering, we introduce a indicator variable $Z_{ij}$, s.t.~for any $i,j \in V$
\[
    Z_{ij} = \begin{cases}
        1 & \text{if } \exists P_\ell \in \mathcal{P} \text{ s.t. } i,j \in P_\ell\\
        0 & \text{otherwise}
    \end{cases}
\]

In other words, $Z_{ij} = 1$ if vertex $i$ and $j$ are in the same cluster, otherwise $Z_{ij} = 0$. Observe that the probability of vertex $i$ and $j$ being in the same cluster is $\frac{1}{4}$ (having placed vertex $i$ in one of the 4 clusters, there is a $\frac{1}{4}$ chance that we also place $j$ in the same cluster). With this, we can determine the expected value of $Z_{ij}$:
\[
    \mathbb{E}[Z_{ij}] = \text{Pr}(\exists P_\ell \in \mathcal{P} \text{ s.t. }i,j \in P_\ell) = \frac{1}{4}
\]
By using the above observation and linearity of expectation, we can compute the expected value of the objective function as follows:

\begin{align*}
    \mathbb{E}&\left[ \sum_{\{i,j\} \in E(\mathcal{P})} w_{ij}^+ + \sum_{\{i,j\} \in E \setminus E(\mathcal{P})} w_{ij}^-\right]\\
    &= \mathbb{E}\left[ \sum_{\{i,j\} \in E} w_{ij}^+ Z_{ij} + \sum_{\{i,j\} \in E} w_{ij}^- (1 - Z_{ij})\right]\\
    &= \sum_{\{i,j\} \in E} w_{ij}^+ \mathbb{E}[Z_{ij}] + \sum_{\{i,j\} \in E} w_{ij}^- (1 - \mathbb{E}[Z_{ij}])\\
    &= \frac{1}{4}\sum_{\{i,j\} \in E} w_{ij}^+ + \frac{3}{4}\sum_{\{i,j\} \in E} w_{ij}^-
\end{align*}

In code, this corresponds to following computation:
\begin{lstlisting}
1/4 * np.sum(W_plus) + 3/4 * np.sum(W_minus)
\end{lstlisting}
For the instance \texttt{denmark-0.6.txt}, we obtain an expected value of $1520.01$.\vspace{1em}

We use randomized rounding on the above SDP to find a 4-clustering. We partition the vertices into 4 sets by separating the hyperspace using two independent random hyperplanes $r_1, r_2 \in \mathbb{R}^n$, i.e.
\begin{align*}
    P_1 &= \{i \in V\ |\ r_1 \cdot v_i \ge 0,\ r_2 \cdot v_i \ge 0 \}\quad P_2 = \{i \in V\ |\ r_1 \cdot v_i \ge 0,\ r_2 \cdot v_i < 0 \}\\
    P_3 &= \{i \in V\ |\ r_1 \cdot v_i < 0,\ r_2 \cdot v_i \ge 0 \}\quad P_4 = \{i \in V\ |\ r_1 \cdot v_i < 0,\ r_2 \cdot v_i < 0 \}
\end{align*}
In code, we compute random partitions and evaluate their weight as follows:

\begin{lstlisting}
partition_weights = []
partitions = []
repetitions = 10**5

for _ in range(repetitions):
  R = np.random.normal(size = (2, n))
  R_dot_V = np.dot(R, V)
  
  P = [(R_dot_V[0] >= 0) * (R_dot_V[1] >= 0), 
       (R_dot_V[0] >= 0) * (R_dot_V[1] < 0), 
       (R_dot_V[0] < 0) * (R_dot_V[1] >= 0),
       (R_dot_V[0] < 0) * (R_dot_V[1] < 0)]
  
  partitions.append([np.array(words)[c] for c in P])
  
  plus_edges = np.sum([np.outer(c, c) for c in P], axis=0)
  minus_edges = 1 - plus_edges  
  
  partition_weights.append(np.sum(W_plus * plus_edges + W_minus * minus_edges))
\end{lstlisting}

\begin{itemize}
    \item Lines 1-5: As we repeat the randomized rounding procedure multiple times (Monte Carlo Algorithm), we define the number of repetitions (\lstinline|repetitions|) and initialize variables to store the weights (\lstinline|partition_weights|) of the resulting partitions and the actual partitions of the words (\lstinline|partitions|).
    \item Lines 6-7: We compute two independent random hyperplanes and the dot product between the hyperplanes and the vectors.
    \item Lines 9-12: Using the above computed dot product, we compute the four partitions and store each cluster as a boolean array.
    \item Line 14: We compute the words in each cluster and add them to \lstinline|partitions|.
    \item Lines 16-17: We determine the edges that are inside the sets of the partition (\lstinline|plus_edges|) using the outer product of each cluster's boolean array with itself. Note that we assume that the graph is a complete graph. Furthermore, we compute the edges that are between the sets of the partition (\lstinline|minus_edges|), which are just all edges outside of \lstinline|plus_edges|.
    \item Line 19: We compute the total weight of the resulting partition and add it to \lstinline|partition_weights|.
\end{itemize}

Using \lstinline|max| as well as \lstinline|np.argmax| we can extract the weight of the best partition and its corresponding clusters of words from \lstinline|partition_weights| and \lstinline|partitions|. For the instance \texttt{denmark-0.6.txt}, we obtained in one run an objective function value of $1623.07$ for following partition:
\begin{itemize}
    \item Cluster 1: \texttt{'brisbane' 'newcastle' 'melbourne' 'nz' 'australia' 'wales' 'london'
 'sydney' 'cardiff' 'england'}
    \item Cluster 2: \texttt{'switzerland' 'belgium' 'finland' 'slovakia' \\'holland' 'zealand' 'russia'
 'bulgaria' 'czech' 'serbia' \\'denmark' 'austria' 'lithuania' 'germany'
 'slovenia' 'italy' \\'norway' 'ireland' 'croatia' 'brazil' 'spain' 'greece'
 \\'sweden' 'netherlands' 'latvia' 'romania' 'poland' 'europe' \\'hungary'
 'iceland' 'belarus' 'azerbaijan' 'kazakhstan' \\'estonia' 'morocco'}
\end{itemize}







\end{document}
